\section{Introduction}
The current web and mobile development forces developers to react to new problems, such as multiuser and high concurrency requests. One popular client side scripting language is JavaScript. With Node Js it was possible to create JavaScript code outside of the browser environment and therefore open it up to server side development. Node Js is based on the V8 engine developed by Google and allows for fast scalable web applications \cite{NodeJs}. Node Js uses an event-driven, non-blocking I/O which makes it perfect for scalable web applications which support high concurrency \cite{lei2014performance}. 

Another option to overcome this is by outsourcing your network and backend to a cloud provider, by using services to get the desired functionality. This allows the customer to be certain the functionality is always given (over 99\% \cite{FirebaseSLA}) and it is only billed for the usage.

The aim of this paper is to compare both options in different categories and show the advantages and disadvantages for each, and which usecases fit them best.

